\begin{itemize}
\item[(1)]
$$A = \begin{bmatrix}
1 & 2 \\
3 & 4
\end{bmatrix} \rightarrow \begin{bmatrix}
1 & 2 \\
0 & -2
\end{bmatrix} \rightarrow \begin{bmatrix}
1 & 2 \\
0 & 1
\end{bmatrix} \rightarrow \begin{bmatrix}
1 & 0 \\
0 & 1
\end{bmatrix}$$
$$\rightarrow A = \begin{bmatrix}
1 & 0 \\
3 & 1
\end{bmatrix}\begin{bmatrix}
1 & 0 \\
0 & -2
\end{bmatrix}\begin{bmatrix}
1 & 2 \\
0 & 1
\end{bmatrix}$$
We must now show that there does not exist a pair of elementary matrices $E_1, E_2$ such that $A = E_1E_2$, or equivalently $E_1^{-1}A = E_2$. Let $E$ be the set of all elementary matrices. We will proceed by brute force: suppose 
$$E_1 = \begin{bmatrix}
1 & c \\
0 & 1
\end{bmatrix} \rightarrow E_1^{-1} = \begin{bmatrix}
1 & -c \\
0 & 1
\end{bmatrix}$$
where $c \neq 0$. Then
$$E_1^{-1}A = \begin{bmatrix}
1 & -c \\
0 & 1
\end{bmatrix}\begin{bmatrix}
1 & 2 \\
3 & 4
\end{bmatrix} = \begin{bmatrix}
1 - 3c & 2 - 4c \\
3 & 4
\end{bmatrix} \not \in E$$
Suppose
$$E_1 = \begin{bmatrix}
1 & 0 \\
c & 1
\end{bmatrix} \rightarrow E_1^{-1} = \begin{bmatrix}
1 & 0 \\
-c & 1
\end{bmatrix}$$
where $c \neq 0$. Then
$$E_1^{-1}A = \begin{bmatrix}
1 & 0 \\
-c & 1
\end{bmatrix}\begin{bmatrix}
1 & 2 \\
3 & 4
\end{bmatrix} = \begin{bmatrix}
1 & 2 \\
3 - c & 4 - 2c
\end{bmatrix} \not \in E$$
Suppose
$$E_1 = \begin{bmatrix}
0 & 1 \\
1 & 0
\end{bmatrix} \rightarrow E_1^{-1} = \begin{bmatrix}
0 & 1 \\
1 & 0
\end{bmatrix}$$
Then
$$E_1^{-1}A = \begin{bmatrix}
0 & 1 \\
1 & 0
\end{bmatrix}\begin{bmatrix}
1 & 2 \\
3 & 4
\end{bmatrix} = \begin{bmatrix}
3 & 4 \\
1 & 2
\end{bmatrix} \not \in E$$
Suppose
$$E_1 = \begin{bmatrix}
c & 0 \\
0 & 1
\end{bmatrix} \rightarrow E_1^{-1} = \begin{bmatrix}
1/c & 0 \\
0 & 1
\end{bmatrix}$$
where $c \neq 0, 1$. Then
$$E_1^{-1}A = \begin{bmatrix}
1/c & 0 \\
0 & 1
\end{bmatrix}\begin{bmatrix}
1 & 2 \\
3 & 4
\end{bmatrix} = \begin{bmatrix}
1/c & 2/c \\
3 & 4
\end{bmatrix} \not \in E$$
Suppose
$$E_1 = \begin{bmatrix}
1 & 0 \\
0 & c
\end{bmatrix} \rightarrow E_1^{-1} = \begin{bmatrix}
c & 0 \\
0 & 1/c
\end{bmatrix}$$
where $c \neq 0, 1$. Then
$$E_1^{-1}A = \begin{bmatrix}
1 & 0 \\
0 & 1/c
\end{bmatrix}\begin{bmatrix}
1 & 2 \\
3 & 4
\end{bmatrix} = \begin{bmatrix}
1 & 2 \\
3/c & 4/c
\end{bmatrix} \not \in E$$
\item[(2)]
Let
$$A = \begin{bmatrix}
0 & -1 \\
1 & 0
\end{bmatrix}$$
Then
$$A^2 = \begin{bmatrix}
0 & -1 \\
1 & 0
\end{bmatrix}\begin{bmatrix}
0 & -1 \\
1 & 0
\end{bmatrix} = \begin{bmatrix}
-1 & 0 \\
0 & -1
\end{bmatrix} = -I$$
So, the complex number $a + bi$ can be represented by the matrix
$$\begin{bmatrix}
a & -b \\
b & a
\end{bmatrix}$$
Then, we can add two complex numbers $(a + bi) + (c + di) = (a + c) + (b + d)i$:
$$\begin{bmatrix}
a & -b \\
b & a
\end{bmatrix} + \begin{bmatrix}
c & -d \\
d & c
\end{bmatrix} = \begin{bmatrix}
a + c & -(b + d) \\
b + d & a + c
\end{bmatrix}$$
And we can multiply two complex numbers $(a + bi)(c + di) = (ac - bd) + (ad + bc)i$:
$$\begin{bmatrix}
a & -b \\
b & a
\end{bmatrix}\begin{bmatrix}
c & -d \\
d & c
\end{bmatrix} = \begin{bmatrix}
ac - bd & -(ad + bc) \\
ad + bc & ac - bd
\end{bmatrix}$$
\item[(3)]
\begin{itemize}
\item[(a)]
$$\det\begin{bmatrix}
1 & 1 & 1 \\
a & b & c \\
a^2 & b^2 & c^2
\end{bmatrix} = \det\begin{bmatrix}
b & c \\
b^2 & c^2
\end{bmatrix} - \det\begin{bmatrix}
a & c \\
a^2 & c^2
\end{bmatrix} + \det\begin{bmatrix}
a & b \\
a^2 & b^2
\end{bmatrix}$$
$$= (bc^2 - b^2c) - (ac^2 - a^2c) + (ab^2 - a^2b) = c(bc  - b^2 + a^2 - ac) + ab(b - a)$$
$$= c((a - b)(a + b) + c(b - a)) + ab(b - a) = -c(b - a)(a + b - c) + ab(b - a)$$
$$= (b - a)(ab - ac - bc + c^2) = (b - a)(a(b - c) - c(b - c)$$
$$= (b - a)(b - c)(a - c) = (b - a)(c - a)(c - b)$$
\item[(b)]
Let 
$$A = \begin{bmatrix}
1 & 1 & 1 & \cdots & 1 \\
a_1 & a_2 & a_3 & \cdots & a_k \\
a_1^2 & a_2^2 & a_3^2 & \cdots & a_k^2 \\
\vdots & \vdots & \vdots & \ddots & \vdots \\
a_1^{k-1} & a_2^{k-1} & a_3^{k-1} & \cdots & a_k^{k-1}
\end{bmatrix}$$
I claim that $\det A = \prod_{i < j}(a_j - a_i)$. Suppose $n = 2$. Then $\det A = (a_2 - a_1)$, so the statement is valid for $n = 2$. Suppose the statement is true for $n = k - 1$. Then for $n = k$,
$$\det A = \det\begin{bmatrix}
1 & 1 & 1 & \cdots & 1 \\
a_1 & a_2 & a_3 & \cdots & a_k \\
a_1^2 & a_2^2 & a_3^2 & \cdots & a_k^2 \\
\vdots & \vdots & \vdots & \ddots & \vdots \\
a_1^{k-1} & a_2^{k-1} & a_3^{k-1} & \cdots & a_k^{k-1}
\end{bmatrix}$$ 
$$= \det\begin{bmatrix}
0 & 1 - a_2/a_1 & 1 - a_3/a_1 & \cdots & 1 - a_k/a_1 \\
0 & a_2 - a_2^2/a_1 & a_3 - a_3^2/a_1 & \cdots & a_k - a_k^2/a_1 \\
0 & a_2^2 - a_2^3/a_1 & a_3^2 - a_3^3/a_1 & \cdots & a_k^2 - a_k^3/a_1 \\
\vdots & \vdots & \vdots & \ddots & \vdots \\
0 & a_2^{k-2} - a_2^{k-1}/a_1 & a_3^{k-2} - a_3^{k-1}/a_1 & \cdots & a_k^{k-2} - a_k^{k-1}/a_1 \\
a_1^{k-1} & a_2^{k-1} & a_3^{k-1} & \cdots & a_k^{k-1}
\end{bmatrix}$$
$$= \frac{1}{a_1^{k-1}}\det\begin{bmatrix}
0 & a_1 - a_2 & a_1 - a_3 & \cdots & a_1 - a_k \\
0 & a_2(a_1 - a_2) & a_3(a_1 - a_3) & \cdots & a_k(a_1 - a_k) \\
0 & a_2^2(a_1 - a_2) & a_3^2(a_1 - a_3) & \cdots & a_k^2(a_1 - a_k) \\
\vdots & \vdots & \vdots & \ddots & \vdots \\
0 & a_2^{k-2}(a_1 - a_2) & a_3^{k-2}(a_1 - a_3) & \cdots & a_k^{k-2}(a_1 - a_k) \\
a_1^{k-1} & a_2^{k-1} & a_3^{k-1} & \cdots & ^{k-1}
\end{bmatrix}$$
$$= \frac{(a_1 - a_2)(a_1 - a_3)...(a_1 - a_k)}{a_1^{k-1}}\det\begin{bmatrix}
0 & 1 & 1 & \cdots & 1 \\
0 & a_2 & a_3 & \cdots & a_k \\
0 & a_2^2 & a_3^2 & \cdots & a_k^2 \\
\vdots & \vdots & \vdots & \ddots & \vdots \\
0 & a_2^{k-2} & a_3^{k-2} & \cdots & a_k^{k-2} \\
a_1^{k-1} & a_2^{k-1} & a_3^{k-1} & \cdots & a_k^{k-1}
\end{bmatrix}$$
$$= (-1)^{k-1}(a_1 - a_2)(a_1 - a_3)...(a_1 - a_k)\det\begin{bmatrix}
1 & 1 & \cdots & 1 \\
a_2 & a_3 & \cdots & a_k \\
a_2^2 & a_3^2 & \cdots & a_k^2 \\
\vdots & \vdots & \ddots & \vdots \\
a_2^{k-2} & a_3^{k-2} & \cdots & a_k^{k-2} \\
a_2^{k-1} & a_3^{k-1} & \cdots & a_k^{k-1}
\end{bmatrix}$$
$$= \prod_{i < j}(a_j - a_i)$$
\end{itemize}
\item[(4)]
$X$ need not exist if $m > n$. For instance, let
$$A = \begin{bmatrix}
3 \\
0
\end{bmatrix}, B = \begin{bmatrix}
3 \\
1
\end{bmatrix}$$
Then $A$ has a left inverse
$$A' = \begin{bmatrix}
\frac{1}{3} & 0
\end{bmatrix}$$
Then
$$X = \begin{bmatrix}
\frac{1}{3} & 0
\end{bmatrix}\begin{bmatrix}
3 \\
1
\end{bmatrix} = [1]$$
But clearly
$$AX = \begin{bmatrix}
3 \\
0
\end{bmatrix}[1] = \begin{bmatrix}
3 \\
0
\end{bmatrix} \neq B$$
Note that if $m = n$ and $A$ has a left inverse, then $A$ also has a right inverse, and so the procedure is valid.
\item[(5)]
\begin{itemize}
\item[(a)]
Consider $A_1 = (1, 0)$ and $A_2 = (0, 1)$. Then
$$A = \begin{bmatrix}
1 & 0 \\
0 & 1
\end{bmatrix}$$
Clearly, $\det A = \text{area } P = 1$. Now consider arbitrary $A_1, A_2$. 

Suppose without loss of generality that $A_1 \mapsto A_1 + cA_2$ for some $c \neq 0$. Since $A_1$ is translated parallel to $\overline{A_2}$, then the area of the parallelogram remains constant. Accordingly, the operation corresponds to an elementary operation of the first kind, so the determinant remains unchanged.

Suppose $A_1 \mapsto A_2$ and $A_2 \mapsto A_1$. Clearly, the area remains unchanged. Accordingly, the operation corresponds to an elementary operation of the second kind, so the determinant changes by a factor of $-1$.

Suppose without loss of generality that $A_1 \mapsto cA_1$ for some $c \neq 0$. Let $\theta$ be the angle between $\overline{A_1}$ and $\overline{A_2}$. Since $|\sin\theta|$ remains constant, then the area changes by a factor of $|c|$. Accordingly, the operation corresponds to an elementary operation of the third kind, so the determinant changes by a factor of $c$.

Since we can arrive at any $A_1, A_2$ by applying a series of these operations to $(1, 0)$ and $(0, 1)$, then $|\det A| = \text{area }P$.
\item[(b)]
Consider $A_1 = (1, 0, ..., 0)$, $A_2 = (0, 1, 0, ..., 0)$, ..., $A_n = (0, ..., 0, 1)$. Then $A = I_n$. Clearly, $\det A = \text{vol } P = 1$. Now consider arbitrary $A_1, A_2, ..., A_n$. 

Suppose without loss of generality that $A_1 \mapsto A_1 + cA_2$ for some $c \neq 0$. Since $A_1$ is translated parallel to $\overline{A_2}$, then the volume of the parallelepiped remains constant. Accordingly, the operation corresponds to an elementary operation of the first kind, so the determinant remains unchanged.

Suppose withou loss of generality that $A_1 \mapsto A_2$ and $A_2 \mapsto A_1$. Clearly, the area remains unchanged. Accordingly, the operation corresponds to an elementary operation of the second kind, so the determinant changes by a factor of $-1$.

Suppose without loss of generality that $A_1 \mapsto cA_1$ for some $c \neq 0$. Let $\theta_i$ be the angle between $\overline{A_1}$ and $\overline{A_i}$, for $i \neq 1$. Since $|\sin\theta_i|$ remains constant, then the volume changes by a factor of $|c|$. Accordingly, the operation corresponds to an elementary operation of the third kind, so the determinant changes by a factor of $c$.

Since we can arrive at any $A_1, A_2, ..., A_n$ by applying a series of these operations to $(1, 0, ..., 0)$, ..., $(0, ..., 0, 1)$, then $|\det A| = \text{vol }P$.
\end{itemize}
\item[(6)]
\begin{itemize}
\item[(a)]
We will proceed by means of induction: suppose $A$ is a $1 \times 1$ matrix. Then clearly $A = LU$, where $L = A$ and $U = I_1$, and also $L$ and $U$ are unique. Suppose for all $k - 1 \times k - 1$ matrices $A'$, there exists unique $L'$ and $U'$ such that $A' = L'U'$. Now consider the $k \times k$ matrix $A$. We can express $A$ as
$$A = \begin{bmatrix}
\begin{array}{c|c}
A' & a \\
\hline
b & c
\end{array}
\end{bmatrix}$$
where $A'$ has dimensions $k - 1 \times k - 1$, $a$ has dimensions $k - 1 \times 1$, $b$ has dimensions $1 \times k - 1$, and $c$ has dimensions $1 \times 1$. I claim that if $A = LU$, then $L$ and $U$ are unique. Decomposing $L$ and $U$ into submatrices with the same dimensions as the submatrices of $A$, we can write
$$A = LU = \begin{bmatrix}
\begin{array}{c|c}
L' & d \\
\hline
e & f
\end{array}
\end{bmatrix}\begin{bmatrix}
\begin{array}{c|c}
U' & g \\ 
\hline
h & i
\end{array}
\end{bmatrix} = \begin{bmatrix}
\begin{array}{c|c}
L'U' + dh & L'g + di \\
\hline
eU' + fh & eg + fi
\end{array}
\end{bmatrix}$$
$$\begin{bmatrix}
\begin{array}{c|c}
L'U' & L'g \\
\hline
eU' & eg + fi
\end{array}
\end{bmatrix} = \begin{bmatrix}
\begin{array}{c|c}
A' & a \\
\hline
b & c
\end{array}
\end{bmatrix}$$
Note that $L'$ is lower triangular and $U'$ is upper triangular only 1s on its diagonal. Thus by the inductive hypothesis, $L'$ and $U'$ are unique. It now suffices to show that $g, e, f, i$ are unique.

Consider $L'g = a$. Consider the 1st row of $a$: $a_1 = L'_1g = L'_{11}g_1 \rightarrow g_1 = \frac{a_1}{L'_{11}}$. Suppose we can uniquely determine $g_i$ for $i < k$. For $i = k$, then $a_k = L'_kg = \sum_{j=1}^k L'_{kj}g_j \rightarrow g_k = \frac{1}{L'_{kk}}(a_k - \sum_{j=1}^{k-1}L'_{kj}g_j)$

Consider $eU' = b$. Consider the 1st column of $b$: $b_1 = eU'_1 = e_1U'_{11} \rightarrow e_1 = b_1$. Suppose we can uniquely determine $e_i$, for $i < k$. For $i = k$, then $b_k = eU'_k = \sum_{j=1}^k e_jU'_{jk} \rightarrow e_k = b_k - \sum_{j=1}^{k-1} e_jU'_{jk}$.

Consider $eg + fi = c$. Since $U_{x} = 1$ for all $x$, then $i = 1$. So, $f = c - eg = c - \sum_{i=1}^{k-1}e_1g_1$. By the uniqueness of $e$ and $g$ above, $f$ is then unique.

Since we have found unique $g, e, f, i$, then $L$ and $U$ are unique.
\item[(b)]
From part (a), then for $i, j$ where $i > j$ we have the following recursive formulae:
$$\ell_{11} = a_{11}$$
$$u_{ii} = 1$$
$$\ell_{ji} = u_{ij} = 0$$
$$\ell_{ij} = a_{ij} - \sum_{k=1}^{j-1} \ell_{ik}u_{kj}$$
$$u_{ji} = \frac{1}{\ell_{jj}}\left( a_{ji} - \sum_{k=1}^{j-1} \ell_{jk}u_{ki}\right)$$
$$\ell_{ii} = a_{ii} - \sum_{k=1}^{i-1}\ell_{ik}u_{ki}$$
Solving these formulae allows us to compute $L$ and $U$.
\item[(c)]
Consider the $m \times n$ matrix $A$. Define $A'$ to be the row-permuted form of $A$, where there eixsts a series of permutations such that we can transform $A'$ into $A"$ holding these properties:
\begin{itemize}
\item[1.] The first nonzero entry in every row is 1. This is a pivot
\item[2.]The first nonzero entry of row $i + 1$ is to the right of the first nonzero entry of $i$.
\end{itemize} 

We can use the following procedure to row reduce $A$ into a matrix $A'$:

If $n = 1$, then normalize the matrix using a Type 3 operation.

If $m = 1$, then find the first row containing a nonzero entry, normalize the entry using a Type 3 operation, and clear out the entries below that row using Type 1 operations

Find the first column that contains a nonzero entry. Find the first row in that column that contains a nonzero entry: denote this entry $a_{ij}$. Normalize this entry using a Type 3 operation. Then clear out the entries in column $j$ below row $i$ using Type 1 operations. Now inductively row reduce $A_{ij}$ to $A'_{ij}$: ie. row reduce the matrix $A$ without all columns $\leq j$, and row $i$.

Each Type 1 operation is a lower triangular matrix, and each Type 3 operation is a diagonal matrix. Therefore, letting $L$ be the sequence of Type 1 and Type 3 operations we have performed, we have $LA = A'$. Since $U = A" = PA'$ for some permutations $P$, then we have $PLA = U \rightarrow A = L^{-1}P^{-1}U$. It is easy to see that $L^{-1}$ is also a lower triangular matrix (since the inverse of a Type 1 operation that is lower triangular is also lower triangular and the inverse of a Type 3 operation is also diagonal), and the inverse of a permutation matrix is also a permutation matrix. Furthermore, since $A$ is invertible, then every column of $A"$ has a pivot, and we can transform $A"$ into the identity using a series of upper triangular Type 3 operations. Therefore, we have $A = LPU$ for a lower triangular $L$, a permutation $P$, and an upper triangular $U$.
\end{itemize}
\item[(7)]
\begin{itemize}
\item[(a)]
If $\det A \neq 0$, then $A$ is invertible. So, we have
$$X = A^{-1}B$$
Since by Cramer's Rule $A^{-1} = \frac{1}{\det A}\text{adj }A$, and since $\text{adj }A$ has integer entries (since each $\det A_{ij}$ is an integer), then $A^{-1}$ has rational entries. Since $B$ also has integer entries, then $X$ has rational entries.
\item[(b)]
Consider
$$A = \begin{bmatrix}
2 & 2 \\
0 & 1
\end{bmatrix}, B = \begin{bmatrix}
3 \\
1
\end{bmatrix}$$
Then necessarily
$$X = \begin{bmatrix}
\frac{1}{2} \\
1
\end{bmatrix}$$
$X$ is a rational solution, but there are no additional solutions, so there is not an additional integer solution.
\end{itemize}
\item[(8)]
Suppose $C = I_m - AB$ is invertible. Let $D = I_n - BA$. Consider $E = (I_n + BC^{-1}A)$. Then
$$DE = (I_n - BA)(I_n + BC^{-1}A) = I_n - BA + BC^{-1}A - BABC^{-1}A$$
$$= I_n - BA + B(C^{-1} - ABC^{-1})A = I_n - BA + B((I_m - AB)C^{-1})A$$
$$= I_n - BA + B(CC^{-1})A = I_n - BA + BA = I_n$$
Similarly,
$$ED = (I_n + BC^{-1}A)(I_n - BA) = I_n + BC^{-1}A - BA - BC^{-1}ABA$$
$$= I_n - BA + B(C^{-1} - C^{-1}AB)A = I_n - BA + B(C^{-1}(I_m - AB))A$$
$$= I_n - BA + B(C^{-1}C)A = I_n - BA + BA = I_n$$
Therefore, $D$ is invertible. We can similarly show that if $D$ is invertible, then $C$ is also invertible.
\end{itemize}